\documentclass[12pt, man, dvipsnames, floatsintext]{apa7}      
% Other options: 

%jou (default): Formats the document with an appearance resembling a printed APA journal (e.g., \textit{Journal of Educational Psychology}. The text is typeset in two-sided, two-column format.
%man: Formats the document in close (if not complete) compliance with the requirements for submission to an APA journal (e.g., title page, double- spacing, etc.).
%stu: Formats the document in close (if not complete) compliance with the requirements for student paper (e.g., title page, double-spacing, etc.).
%doc: Formats the document as a typical LATEX document (one-sided, single- column, etc.)

% floatsintext: In man and stu mode, integrates floats (tables and figures) within the body of the text instead of postponing them until after the reference list.

% tt: In man and stu mode, uses typewriter-like font.
\usepackage[american]{babel}        		% See geometry.pdf to learn the layout options. There are lots.   
              		% ... or a4paper or a5paper or ... 
%\geometry{landscape}                		% Activate for rotated page geometry
%\usepackage[parfill]{parskip}    		% Activate to begin paragraphs with an empty line rather than an indent
%\usepackage{mathptmx}
\usepackage{fontspec}
\setmainfont{Times New Roman}
\usepackage{tabularray}
\usepackage{booktabs}
\usepackage{hyperref}
\usepackage{setspace}

\hypersetup{
    colorlinks=true,
    linkcolor=RoyalBlue,
    filecolor=magenta,      
    urlcolor=MidnightBlue,
    citecolor={RawSienna},
    pdftitle={Comprehensive Synthetic Satellite Forest Canopy Dataset for Height Estimation},
	pdfauthor={Gil Isaac E. Gabriel},
	pdfsubject={},
	pdfkeywords={}
    %pdfpagemode=FullScreen,
    }


\urlstyle{same}
\usepackage{csquotes}
%\usepackage{apacite}

\usepackage{epigraph} %quotes in beginning

\usepackage[style=apa, backend=biber]{biblatex} %you need to biber your file to make sure your citations work!


\addbibresource{SyntheticCanopy.bib} 

%    4 pages, 1.5 spacing = 1500 words
\title{Comprehensive Synthetic Satellite Forest Canopy Dataset for Height Estimation}
\author{Gil Isaac E. Gabriel}
\affiliation{{Department of Geography and Planning, West Chester University of Pennsylvania}} % or \affiliation{{Honors College, West Chester University of Pennsylvania}}
\course{GEO 531: Transportation Planning}
\authornote{\addORCIDlink{Gil Isaac E. Gabriel}{0000-0003-2414-8547}

This project is funded by Grant 2244304 from the National Science Foundation. I thank Yang Tang, Dr. Rongjun Qin, and Dr. Jieun Hur for their guidance throughout the research process.

Correspondence regarding this article should be addressed to Gil Isaac E. Gabriel, 506 Business and Public Management Center, 50 Sharpless Street, West Chester, PA 19383. Email: \href{mailto:isaac@gilisaacgabriel.com}{isaac@gilisaacgabriel.com}.
}

\abstract{Tree canopy height is an increasingly important measure due to its ability to inform climate change mitigation strategies. However, current research on tree canopy height estimation on a global scale has indicated limited predictive accuracy due to insufficient remote sensing data. For instance, current LiDAR data from the Global Ecosystems Dynamics Investigation (GEDI) have low spatiotemporal resolutions and only cover about 4\% of the land surface. Machine Learning (ML) algorithms that predict tree canopy heights rely on LiDAR and satellite imagery data. Therefore, although satellite imagery is comprehensive and has high spatiotemporal resolutions, the lack of comprehensive quality LiDAR data reduces the ability of ML algorithms to produce accurate estimations of tree canopy height globally. Thus, the purpose of this research project is to leverage computer graphics technology to create synthetic environments with accurate tree canopy height measurements. 30,000 3D scenes will be generated using an automated workflow and will simulate diverse environments with tree type and height, leaf density, lighting condition, and seasonal variation as parameters. A multispectral (RGB) image, Digital Surface Model, and Digital Elevation Model are extracted from each synthetic scene. This could serve to train ML algorithms as ground truth and improve the accuracy of tree canopy height estimations. More precise tree canopy height measurements will improve ecological assessments such as global carbon stock assessments, informing climate change mitigation, land use management, and conservation efforts.}

\doublespacing
\begin{document}
\shorttitle{Synthetic Forest Canopy Height Estimation Dataset} % A shortened version of the title (for page headers)

\maketitle

%\renewcommand \contentsname{Table of Contents}
%
%\tableofcontents

  

% \setstretch{1.375} %2 I think for double space. Double-check this

 % or this? \singlespacing, \onehalfspacing and \doublespacing can be used in the document preamble, or within the document body to change spacing in part, or all, of your document; the \setstretch{baselinestretch amount} command sets a custom spacing (via changes to \baselinestretch)—it can also be used in the document preamble;


\hypertarget{literature-review}{%
\section{Literature Review}\label{literature-review}}

Canopy height estimation is crucial for sustainability initiatives and
monitoring ecological changes. Accurate measurements of forest canopy
height provide valuable insights into carbon stock and ecosystem
services, which are essential for climate change mitigation and
conservation efforts. Forests play a vital role in absorbing carbon
dioxide from the atmosphere, making them important carbon sinks.
Accurate tracking of carbon absorption by forests has significant
implications for climate change mitigation, resilience, and global
warming \parencite{Lang2023}. Canopy height is a key measure that
provides information about ecological changes and carbon stock
\parencite{Potapov2021,Tang2024}. \textcite{Potapov2021} demonstrate the
ability of canopy height maps to monitor ecological changes, such as
differences in tree heights between natural forests and suburban areas,
secondary forests, and plantations.

\hypertarget{approaches-to-canopy-height-mapping}{%
\subsection{Approaches to Canopy Height
Mapping}\label{approaches-to-canopy-height-mapping}}

Traditionally, canopy height mapping has been done through field
observations, which are time-consuming and labor-intensive. This method
is limited in scale and cannot be applied globally, compromising
researchers' ability to assess carbon storage on a global scale and
monitor progress towards climate goals and emission targets. To address
this issue, researchers have turned to remote sensing data and Machine
Learning (ML) techniques for global canopy height estimation.
\textcite{Lang2023} produced a global canopy height model by combining
Sentinel-2 optical satellite images with height data from the Global
Ecosystem Dynamics Investigation (GEDI) LiDAR mission. Similarly,
\textcite{Potapov2021} created a global canopy height map using Landsat
imagery fused with GEDI RH95 metric, reserving 10\% of GEDI data for
validation and using Airborne Laser Scanning (ALS) as an independent
reference.

\textbf{Challenges and Limitations}:

Despite the advancements in remote sensing technology, there are still
challenges and limitations in canopy height estimation. GEDI provides
sparse elevation data due to limited positioning and movement, and the
mission only began in 2019, resulting in a limited collection of
quality, analysis-ready data. In fact, only approximately 4\% of land
cover is captured by GEDI LiDAR \parencite{Lang2023}. Furthermore,
remote sensing data often contains noise from cloud and/or snow cover,
requiring high spatiotemporal resolutions and image patching to produce
comprehensive data products.

The sparse GEDI data has produced uncertainties that are more likely to
affect certain regions over others. Regions in high latitudes such as
Alaska, Yukon, and Tibet have high predictive uncertainty, with the
latter two likely due to local characteristics that
\citeauthor{Lang2023}'s \citeyear{Lang2023} model has not encountered in
training as they lie outside of GEDI coverage.

Current models, such as the Canopy Height model by \textcite{Lang2023},
struggle with height estimation due to factors like the limited amount
of available data for model validation. Improving these models will
require more comprehensive datasets that are inclusive of diverse
environmental features and conditions. Models are likely to inaccurately
estimate canopy heights in regions that are underrepresented in the
training data.

Researchers are working to address the limitations in canopy height
estimation by creating synthetic datasets that can be used for model
calibration. For example, TreeNet3D is a recent dataset generated
through a fully automated process, containing 1,300 instances each of 10
different tree species \parencite{Tang2024}. Such datasets can help
improve the accuracy and generalizability of canopy height estimation
models.

Canopy height estimation is a critical tool for monitoring ecological
changes and assessing carbon stock, making it essential for
sustainability initiatives and climate change mitigation efforts. While
traditional field observations are time-consuming and limited in scale,
modern approaches utilizing remote sensing data and machine learning
techniques have enabled global canopy height mapping. However,
challenges such as sparse elevation data, noise in remote sensing
imagery, and limited training data representing diverse environments
still persist. Ongoing efforts to create comprehensive synthetic
datasets offer promise for improving the accuracy and applicability of
canopy height estimation models in the future.

\hypertarget{methods}{%
\section{Methods}\label{methods}}

General framework.

I utilized Unreal Engine 5.4 to generate a forest canopy scene,
simulating diverse conditions and environments. The scene include
varying tree species, densities, and distributions to represent
realistic forest structures. Next, I captured images in a 512×512 pixel
square at a spatial resolution of 30 m by positioning a virtual camera
in the scene at a constant height of \{\textasciitilde10,000\} m above
ground. From this position, three simulated image types are taken: (a)
optical image in natural RGB color to simulate satellite imagery; (b) a
Depth Map to represent a Digital Terrain Model (DTM); and (c) another
Depth Map simulating a Digital Surface Model (DSM), which adds above
ground elevation data to the DTM. I created a total of 30,000 samples,
amounting to 90,000 images---three images taken for each forest canopy
scene. A compiled set of these images serve as the dataset which could
be used for training and validating Machine Learning (ML) models and
estimated canopy height products.

\hypertarget{d-scene-generation}{%
\subsection{3D Scene Generation}\label{d-scene-generation}}

\hypertarget{procedural-content-generation-pcg-framework}{%
\paragraph{Procedural Content Generation (PCG)
framework}\label{procedural-content-generation-pcg-framework}}

Unreal Engine version 5.4 provides a number of features that allow for
an automatic placement of 3D assets such as trees onto scenes. UE5.4
presents an opportunity to batch-create scenes with varying
configurations and parameters. The Foliage tool in Unreal Engine allows
for the placement of Static Meshes, which are 3D models representing
individual trees. These Static Meshes can be automatically scattered
across a landscape based on customizable rules such as density, scale,
and rotation. Additionally, the Landscape tool enables the creation of
realistic terrain features like hills, valleys, and slopes, providing a
diverse environment for the procedurally generated forest scenes

\emph{provides artists and designers the ability to build fast,
iterative tools and content of any complexity ranging from asset
utilities, such as buildings or} \textbf{\emph{biome generation}}\_, up
to entire worlds.\_

Parameters

\hypertarget{defining-the-landscape}{%
\subsubsection{\texorpdfstring{\textbf{Defining the
landscape}}{Defining the landscape}}\label{defining-the-landscape}}

The chart outlines a detailed methodology for generating procedurally
created forest scenes and rendering their outputs in a 3D scene. The
process begins with defining the landscape using ground texture assets
and elevation data. This landscape foundation is enhanced using the
Procedural Content Generation (PCG) framework, which allows for the
specification of various parameters such as tree types, biomes, leaf and
branch density, time of day, light conditions, and seasonal changes.

Once the parameters are set, 3D tree assets are scattered across the
landscape based on the defined criteria like density, scale, and
rotation. The landscape tool further refines the scene by incorporating
realistic terrain features such as hills, valleys, and slopes, ensuring
the forest scenes are procedurally generated to mirror natural
environments.

Next, the scene is positioned using a virtual camera, preparing it for
image rendering and output. At this stage, the process allows for
updates to the 3D scene by changing parameter values or modifying the
landscape, ensuring flexibility and iterative refinement. The rendered
outputs can be viewed in different modes, including scene depth view
mode, regular lit view mode, and optical ``satellite'' natural RGB color
images. These outputs are organized to correspond to each other,
ensuring consistency across the digital elevation/terrain model, digital
surface model, and optical images.

\begin{APAitemize}
\item
  Height range
\item
  Per biome
\item
  Tree Types
\item
  Biomes

  \begin{APAitemize}
  \item
    Biome-level analysis of canopy heights showing \textbf{distribution
    of heights} (Lang et al., 2023). Will use these height
    specifications to parameterize tree heights in the PCG.
  \end{APAitemize}
\item
  Leaf/Branch Density
\item
  Time
\item
  Light Conditions
\item
  Seasonal Changes
\item
  Spatial Resolution: 30 m

  \begin{APAitemize}
  
  \item
    Grid size (area)
  \item
    Goal: 512*512 pixels
  \end{APAitemize}
\end{APAitemize}

\hypertarget{image-rendering-and-output}{%
\subsection{Image Rendering and
Output}\label{image-rendering-and-output}}

\hypertarget{d-scene-modification}{%
\subsection{3D Scene Modification}\label{d-scene-modification}}

\hypertarget{automation-and-batch-processing}{%
\subsection{Automation and Batch
Processing}\label{automation-and-batch-processing}}


\printbibliography
\end{document}